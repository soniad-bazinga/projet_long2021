\documentclass[a4paper, 11pt, openright, twoside]{article}

\usepackage{multirow}
\usepackage{enumitem}
\usepackage[french]{babel}
\usepackage[left=2cm,right=2cm,top=2cm,bottom=2cm]{geometry}
\frenchbsetup{StandardLists=true} % à inclure si on utilise \usepackage[french]{babel}

\title{Evaluation des besoins du client}
\date{\vspace{-10ex}}
\begin{document}
	
\maketitle

\begin{itemize}[label=\-]
\item $md2htm$l:
	\begin{itemize}[label=-]
		\item Commande \emph{ssg build} qui transforme un fichier \emph{file.md} de variante CommonMark en un fichier \emph{file.html} en HTML5. Le html doit être validé par le validateur vNu dans le processus de test. Par défaut, les fichiers produits sont placés dans le répertoire \emph{\_output/}, ou dans le répertoire \emph{DIR} si celui-ci est spécifié via l’option \emph{--output-dir DIR}.
		\item Un script shell nommé \emph{ssg}, auquel on peut faire appel par la commande \emph{ssg commande [options] [arguments]}, où \emph{[options]} est une liste d’options préfixée par deux tirets dépendant de la commande choisie et où \emph{[arguments]} est une liste d’arguments.
	\end{itemize}

\item $buildsite$ :
	\begin{itemize}[label=-]
		\item La fonctionnalité $buildsite$ est lancée par la commande \emph{ssg build}. Par défaut, le répertoire d’input est le répertoire courant (sauf si l’option\emph{ --input-dir} est utilisée), et les fichiers sont produits dans \emph{\_output/} (sauf si \emph{--output-dir} est utilisée). Elle nécessite la présence de la structure suivante : un fichier de configuration \emph{site.toml}, un sous-répertoire \emph{contents} contenant des fichiers \emph{.md} dont au moins un fichier\emph{ index.md}. La structure est recopiée dans le répertoire d'output et tous les fichiers \emph{.md} sont traduits avec la fonctionnalité $md2html$.
	\end{itemize}

\item $help$ :
	\begin{itemize}[label=-]
		\item 
La commande \emph{ssg help} affiche un message sur la sortie standard documentant les entrées acceptées par \emph{ssg}, elle doit être mise à jour au fur et à mesure des commandes implémentées.
	\end{itemize}

\item $metadata$ :
	\begin{itemize}[label=-]
		\item 
L’outil doit pouvoir lire les metadata dans les fichiers\emph{ .md}, si celles-ci sont en tête de fichier, au format TOML, et comprises entre deux lignes ne contenant que des “++++”. Les associations clef/valeur doivent être stockées pour être utilisées lors de l'exécution. Si la clef a la valeur \emph{title}, \emph{date} ou \emph{draft} il faut s'assurer du bon type de ces dernières (respectivement deux chaînes de caractères et un booléen). S'il y a une métadonnée de clef \emph{draft} dont la valeur est \emph{true} alors notre outil doit ignorer ce fichier \emph{.md}.
	\end{itemize}

\item $static$ :
	\begin{itemize}[label=-]
		\item 
L’outil doit prendre en compte dans l'architecture du site un répertoire \emph{static} contenant des fichiers comme du css, des images, vidéos etc. Il doit copier ce répertoire et son contenu tel quel sans le traiter.
	\end{itemize}

\item $template$ :
	\begin{itemize}[label=-]
		\item 
Dans les métadonnées, les fichiers \emph{.md} peuvent spécifier, via la clef \emph{template}, le chemin vers un fichier texte (par exemple, \emph{exemple.html}) dont le texte sera substitué au fichier \emph{.md}. S'il n'y a pas de clef \emph{template} dans les métadonnées et qu'un fichier \emph{default.html} existe alors il sera automatiquement utilisé comme patron. Les patrons doivent tous se trouver dans le répertoire \emph{template}, lui-même situé dans le répertoire \emph{static}. Les patrons peuvent contenir des motifs compris entre \emph{\{\{ \}\}} qui seront remplacés par la valeur de clef correspondante dans le fichier \emph{.md} (par exemple le motif \emph{metadata.K} est remplacé par la valeur de la clef \emph{K} dans les metadata du fichier \emph{.md} traité). Le motif \emph{include P} est remplacé par le contenu du patron situé au chemin \emph{P} post-substitution des motifs.
\end{itemize}
\end{itemize}

\end{document}