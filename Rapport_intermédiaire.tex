\documentclass{article}

\usepackage[T1]{fontenc}
\usepackage[utf8]{inputenc}
\usepackage[french,english]{babel}

%% This package is necessary to use \includegraphics.
\usepackage{graphicx}

%% This package is necessary to define hyperlinks.
\usepackage{hyperref}

%% These packages are necessary to include code.
\usepackage{listings}
%%\usepackage{minted} % colored

%% This package is needed to enchance mathematical formulas.
\usepackage{amsmath}



% This is a comment line in latex

% Latex allows you to define your own "commands",
% better known as "macros" in the Latex world.
% The following line is an example of such definition.
\newcommand{\latex}{\LaTeX}


% The next lines contain some meta informations about this document.

\title{Rapport intermédiaire}
\author{DAHMANI Sonia, SANTOS DUARTE Elsa}
\date{18.02.2022}
\begin{document}
	
\maketitle

\section{Présentation générale}
	Le but de ce projet est de convertir des images en mélodie. Pour ce faire, le code est principalement divisé en deux parties. D'une part, du code python, permettant l'analyse d'images et leur traduction en données formelles. Et d'autre part, du code java, récupérant ces données formelles et les transformant en code Sonic-pi pour la création de mélodies.
	\subsection{Analyse d'images et production de données formelles}
	\subsection{Traduction de données formelles en mélodie}
		Dans cette partie on souhaite convertir les données formelles telles que les entiers rgb, les formes ou encore leur taille en mélodie. On doit donc établire une correspondance de l'une à l'autre. Les couleurs servent à la hauteur de note : la gamme est traduite depuis la couleur dominante, les accords 
\section{Réalisé jusqu'ici}
	\subsection{Analyse d'images et production de données formelles}
	\subsection{Traduction de données formelles en mélodie}
		La première étape a été d'établir une correspondance couleur/note. Dans un premier temps, et pour faciliter les correspondances, les couleurs on été réduites en intervalle rgb tel que $r,g,b \in [0,2]$. Ainsi, si $r$, $g$ ou $b$ sont compris entre $0$ et $85$, ils seront interprétés comme $0$, s'ils sont compris entre $85$ et $170$, ils sont interprétés comme $1$ et s'ils sont compris entre $170$ et $255$, ils sont interprétés comme $2$. 
\section{Difficultés rencontrées}
	Le projet étant étendu sur deux principales parties, les premières difficultés sont rapidement venues de pair. D'une part, la difficulté d'analyse d'images. \\
	La première étape était de trouver une bibliothèque d'images suffisamment vaste pour pouvoir reconnaître les données necéssaire au projet, pour l'instant les formes. La seconde difficulté était de trouver un moyen efficace d'analyser les images : soit en réduisant le nombre de leurs couleurs, soit en réduisant directement leur qualité. 
	D'autre part de l'analyse d'image, il y a eu les difficultés liées à la production musicale.\\
	La première d'entre elles était de trouver un langage de sortie permettant l'écriture de la musique. Dans un premier temps, les bibliothèques de traitement midi sous java paraissaient une bonne solution. Puis, au fur et à mesure, le langage \emph{Sonic-pi} nous est apparu comme plus intuitif et adapté à la traduction de données formelles en musique.
	La seconde difficulté concernant la traduction de données formelles en musique, a été et est toujours de trouver une représentation du rythme. Ce qu'il nous paraissait à priori le plus logique était de se servir des formes reconnues lors de l'analyse d'image et de les convertir en rythme.
\section{Etapes futures} 

\end{document}